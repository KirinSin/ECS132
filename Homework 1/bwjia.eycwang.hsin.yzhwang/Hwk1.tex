\documentclass[11pt]{article}

\usepackage{amsmath}

\setlength{\oddsidemargin}{0.0in}
\setlength{\evensidemargin}{0.0in}
\setlength{\topmargin}{-0.25in}
\setlength{\headheight}{0in}
\setlength{\headsep}{0in}
\setlength{\textwidth}{6.5in}
\setlength{\textheight}{9.25in}
\setlength{\parindent}{0in}
\setlength{\parskip}{2mm}

\begin{document}

ECS 132 Winter 2016 Homework 1

Justin Jia, Ethan Wang, Ho Lun Sin, Yangzihao Wang

\section{Question 1}

1) There are 3 items in Box 1.

\begin{align*}
P(\text{3 items in box 1}) &= P(W_1 = 1 \text{ and } W_2 = 1 \text{ and } W_3 = 1 \text{ and } W_4 \neq 1)\\
                              &\quad + P(W_1 = 2 \text{ and } W_2 = 1 \text{ and } W_3 = 1)\\
                              &\quad + P(W_1 = 1 \text{ and } W_2 = 2 \text{ and } W_3 = 1)\\
                              &\quad + P(W_1 = 1 \text{ and } W_2 = 1 \text{ and } W_3 = 2)\\
                           &= P(W_1 = 1 \text{ and } W_2 = 1 \text{ and } W_3 = 1 \text{ and } W_4 \neq 1)\\
                              &\quad + 3 \cdot P(W_a = 2 \text{ and } W_b = 1 \text{ and } W_c = 1)\\
                           &= (\frac13 \cdot \frac13 \cdot \frac13 \cdot \frac23) + 3 \cdot (\frac13 \cdot \frac13 \cdot \frac13)\\
                           &= 0.135802469
\end{align*}

2) The total weight in Box 1 is under 4.

\begin{align*}
P(\text{box 1 total weight} < 4) &= P(W_1 + W_2 + W_3 < 4 \text{ and } W_1 + W_2 + W_3 + W_4 > 4)\\
                                    &\quad + P(W_1 + W_2 < 4 \text{ and } W_1 + W_2 + W_3 > 4)\\
                                    &\quad + P(W_1 < 4 \text{ and } W_1 + W_2 > 4)\\
                                 &= P(W_1 + W_2 + W_3 = 3) \cdot P(W_4 > 1)\\
                                    &\quad + P(W_1 + W_2 = 3) \cdot P(W_3 > 1) + P(W_1 + W_2 = 2) \cdot P(W_3 > 2)\\
                                    &\quad + P(W_1 = 3) \cdot P(W_2 > 1) + P(W_1 = 2) \cdot P(W_2 > 2)\\
                                 &= (\frac13 \cdot \frac13 \cdot \frac13) \cdot \frac23 + (2 \cdot \frac13 \cdot \frac13) \cdot \frac23 + (\frac13 \cdot \frac13) \cdot \frac13 + \frac13 \cdot \frac23 + \frac13 \cdot \frac13\\
                                 &= 0.543209877
\end{align*}

3) The weight of the first item placed into Box 2 is 1.

\begin{align*}
P(\text{weight of first item in 2} = 1) &= P(W_n = 1 | W_{box1} = 4)\\
                                        &= P(w_{box1} = 4 \text{ and } W_n = 1)\\
                                        &= (1 - P(w_{box1} < 4)) \cdot P(W_n = 1)\\
                                        &= (1 - 0.543209877)(\frac13)\\
                                        &= 0.152263374
\end{align*}

4) Given that the weight of the first item placed into Box 2 is 1, the probability that the first item in Box 1 was 1.

\begin{align*}
P(W_{box1,1} = 1 | W_{box2,1} = 1) &= P(W_1 = 1 | W_{box1} = 4)\\
                                   &= \frac{P(W_1 = 1 \text{ and } W_{box1} = 4)}{P(W_{box1} = 4)}\\
                                   &= \frac{P(W_1 = 1 \text{ and } (W_2 + W_3 + W_4 = 3 \text{ or } W_2 + W_3 = 3 \text{ or } W_2 = 3))}{1 - P(w_{box1} < 4)}\\
                                   &= \frac{\frac13 \cdot (\frac13 \cdot \frac13 \cdot \frac13 + 2 \cdot \frac13 \cdot \frac13 + \frac13)}{1 - 0.543209877}\\
                                   &= 0.432432433
\end{align*}

\section{Question 2}

1) Jack is at square 0.

\begin{align*}
P(\text{Jack at 0}) &= P(R_{jack} = 6)\\
                    &= P(R_1 = 6 or (R_1 = 1 and R_2 = 5))\\
                    &= P(\frac16 + \frac16 \cdot \frac16)\\
                    &= 0.194444444
\end{align*}

2) Jill has overtaken Jack.

\begin{align*}
P(\text{Jill overtakes}) &= P(R_{Jill} > R_{Jack} + 2)\\
                         &= {P(R_{Jill} = 5 \text{ and } R_{Jack} = 2 ) + P(R_{Jill} = 6 \text{ and } R_{Jack} < 4 ) + P(R_{Jill} = 7 \text{ and } R_{Jack} < 5 )}\\
                            &\quad + P(R_{Jill} = 8 \text{ and } R_{Jack} < 6) + P(R_{Jill} = 9 \text{ and } R_{Jack} < 7)\\
                         &= (\frac{1}{6} + \frac{1}{6} \cdot \frac16) \cdot (\frac16 + \frac16 \cdot \frac16) + (\frac16 + \frac16 \cdot \frac16) \cdot 2(\frac16 + \frac16 \cdot \frac16)\\
                            &\quad + \frac{1}{36} \cdot 3(\frac16 + \frac16 \cdot \frac16) + \frac{1}{36} \cdot 4(\frac16 + \frac16 \cdot \frac16) + \frac{1}{36} \cdot 5(\frac16 + \frac16 \cdot \frac16)\\
                         &= 0.17824074
\end{align*}

3) Neither Jack nor Jill had a bonus roll, if we are told that he is at the same square as Jill.

\begin{align*}
P(\text{No bouns roll}) &= P(R_{Jill} \text{ and } R_{Jack} \text{ has no bonus} \;|\; R_{Jill} = R_{Jack} + 2)\\
                         &= \frac{P(R_{Jill} = R_{Jack} + 2, R_{Jill} \neq 3 \text{ and } R_{Jack} \neq 1)}{P(R_{Jill} = R_{Jack} + 2)}\\
                         &= \frac{P(R_{Jill1} = 4, R_{Jack1} = 2) + P(R_{Jill1} = 5, R_{Jack1} = 3) + P(R_{Jill1} = 6, R_{Jack1} = 4)}{3(\frac{1}{6} + \frac{1}{6} \cdot \frac16) \cdot (\frac16 + \frac16 \cdot \frac16) + 2(\frac{1}{6} \cdot \frac16) \cdot (\frac16 + \frac16 \cdot \frac16) + \frac{1}{36} \cdot \frac{1}{36}}\\
                         &= \frac{\frac16 \cdot \frac16 + \frac16 \cdot \frac16 + \frac16 \cdot \frac16}{\frac{1}{8}}\\
                         &= 0.666666667
\end{align*}

\end{document}
