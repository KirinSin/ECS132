\documentclass[11pt]{article}

\usepackage{amsmath}

\setlength{\oddsidemargin}{0.0in}
\setlength{\evensidemargin}{0.0in}
\setlength{\topmargin}{-0.25in}
\setlength{\headheight}{0in}
\setlength{\headsep}{0in}
\setlength{\textwidth}{6.5in}
\setlength{\textheight}{9.25in}
\setlength{\parindent}{0in}
\setlength{\parskip}{2mm}

\begin{document}

ECS 132 Winter 2016 Homework 2

Justin Jia, Ethan Wang, Ho Lun Sin, Yangzihao Wang

\section{Question 1}

Continous Problem 1 in Hwk1

1) Expected value of the weight of the first item placed into Box 2

\begin{align*}
P(\text{Weight = 1}) = [P(\text{1, 1, 1, 1}) + P(\text{1, 1, 2}) + P(\text{1, 2, 1}) + P(\text{1, 3}) + P(\text{2, 1, 1}) + P(\text{2, 2}) + P(\text{3, 2})]\cdot \frac13\\
					 = (\frac{1}{81} + \frac19 + \frac 13)\cdot \frac31\\
					 = \frac{37}{243}

P(\text{Weight = 2}) = P(\text{Weight = 1}) + P(\text{1, 1, 1, 2}) + P(\text{1, 2, 2}) + P(\text{2, 1, 2}) + P(\text{3, 2})\\
					 = \frac{85}{243}

P(\text{Weight = 3}) = P(\text{Weight = 2}) + P(\text{1, 1, 3}) + P(\text{2, 3}) + P(\text{3, 3})\\
					 = \frac{121}{243}

E(\text{Weight of the first item placed into Box 2}) = P(\text{Weight = 1}) + 2 \cdot P(\text{Weight = 2}) + 3 \cdot P(\text{Weight = 3}) = 2.345679
\end{align*}

2) Variance of the weight
\begin{align*}

Var(\text{Weight}) = E(W^2) - EW^2\\
				  &= P(\text{Weight = 1}) + 4 \cdot P(\text{Weight = 2}) + 9 \cdot P(\text{Weight = 3}) - 2.345679^2\\
				  &= 6.032922 - 5.502209 = 0.530711

\end{align*}
\section{Question 2}

Bus Ridership Example: Find Cov(L1,L2)

\begin{align*}
Cov(L_1, L_2) &= E(L_1 \cdot L_2) - EL_1 \cdot EL_2\\

E(L_1) = 0.4 + 0.2 \cdot 1 = 0.6

P(L_2 = 1) = 0.5 \cdot 0.4 + 0.4 \cdot 0.8 \cdot 0.5 + 0.4 \cdot 0.4 \cdot 0.2 + 0.1 \cdot 0.2 \cdot 0.8 \cdot 0.5 \cdot 2 + 0.1 \cdot 0.2 \cdot 0.2 \cdot 0.4\\
	= 0.4096 

P(L_2 = 2) = 0.5 \cdot 0.1 + 0.4 \cdot 0.8 \cdot 0.4 + 0.4 \cdot 0.2 \cdot 0.1 + 0.1 \cdot 0.2 \cdot 0.8 \cdot 0.4 \cdot 2 + 0.1 \cdot 0.2 \cdot 0.2 \cdot 0.1 + 0.1 \cdot 0.8 \cdot 0.8 \cdot 0.5\\
	= 0.2312

P(L_2 = 3) = 0.4 \cdot 0.8 \cdot 0.1 + 0.1 \cdot 0.8 \cdot 0.8 \cdot 0.4 + 0.1 \cdot 0.8 \cdot 0.2 \cdot 0.1 \cdot 2\\
	= 0.0608

P(L_2 = 4) = 0.1 \cdot 0.8 \cdot 0.8 \cdot 0.1 = 0.0064

P(L_2) = 1.08

P(L_1 \cdot L_2 = 1) = P(L_1 = 1, L_2 = 1) = 0.192

P(L_1 \cdot L_2 = 2) = P(L_1 = 1, L_2 = 2) +  P(L_1 = 2, L_2 = 1) = 0.1536

P(L_1 \cdot L_2 = 3) = P(L_1 = 1, L_2 = 3) = 0.0356

P(L_1 \cdot L_2 = 4) = P(L_1 = 2, L_2 = 2) = 0.0004

P(L_1 \cdot L_2 = 6) = P(L_1 = 2, L_2 = 3) = 0.0288

P(L_1 \cdot L_2 = 8) = P(L_1 = 2, L_2 = 4) = 0.0064

E(L_1 \cdot L_2) = 1

Cov(L_1, L_2) = 1- 0.6 \cdot 1.08 = 0.352

\end{align*}

\section{Question 3}

\begin{align*}

Var(X+Y) = E[(X+Y)^2] - [E(X+Y)]^2\\
		&= E(X^2 + Y^2 - 2XY) - EX^2 - EY^2 - 2EX \cdot EY\\
		&= Var(X) - Var(Y) - 2EX \cdot EY - 2E(XY)\\
		&=p(1-p) - q(1-q) - 2pq - 2r\\
		&=p - q - p^2 + q^2 - 2pq - 2r

\end{align*}

\section{Question 4}

\begin{align*}

P(D_4 = 1) = \frac12

P(D_4 = 2) = \frac12

E(D_4) = 1

\end{align*}
\end{document}
